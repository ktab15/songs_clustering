\tytul{Środa}{sł. i muz. M. Kamper}{Słodki Całus od Buby}
\begin{textn}

Wstaję wcześnie rano, nowy witam dzień,\\
Budzik zaraz dostał w łeb bo szczekał jak zły pies,\\
Kawa i papieros i wychodzić czas,\\
Idę poprzez smutne miasto, które też by chciało spać.\\
Nagle ktoś wyciąga ręce, krzyczy: Kopę lat!\\
Co u ciebie, pożycz stówę! Mówię: Nie. On: Jesteś cham!\\
Lecz już łapie mnie buddysta, Hare hare hare kriszna\\
Odrzuć mięso, ogól głowę albo wspomóż święty cel!\\
Powiedziałem: Włosy – owszem, lecz ja lubię dobrze zjeść.\\
Rzekł: W następnym swym wcieleniu\\
na sto procent będziesz psem!\\
Oto już następny idzie, ten co wie jak zbawić świat,\\
Bo licencję ma od Boga na dzierżawę nieba bram.\\
On wie, czego mi potrzeba, jaki sens jest w życiu mym,\\
Bo z Jezusem był na studiach, a z Mojżeszem jest na „ty”.

Ja myślę sobie: Co to jest? Co ja tutaj robię?\\
Jak, no jak, jak wytrzymać taki dzień może zwykły człowiek?\\
Myślę o tym, jak cudownie było jeszcze wczoraj,\\
Co dziś jeszcze zdarzy się? Byle do wieczora!

Wieczorem do knajpy, zarobić na chleb,\\
Gęsty dym rozgarniam ręką szukam sceny – jest!\\
Naród siedzi, w rączki klaszcze albo gada żywo,\\
Szef powiada: Forsy nie mam, ale stawiam piwo!\\
Chwila przerwy – krótki oddech złapać chcemy, ale gdzie tam!\\
Siada jeden, tak na oko – pijany poeta,\\
Mówi: Czuję jak nieznośna lekkość głazy w duszy spiętrza...\\
Po czym wybiegł gdzieś za rogiem\\
kontemplować własne wnętrza.

Ja myślę sobie: Co to jest? Co ja tutaj robię?\\
Jak, no jak, jak wytrzymać taki dzień może zwykły człowiek?\\
Myślę o tym, czy mnie wieczór ten znowu czymś zaskoczy\\
Co dziś jeszcze zdarzy się? Byle do północy!

Potem znów jedziemy stały mocny set,\\
Nagle jakiś gościu wstaje, mówi tak: Gracie OK,\\
Mówi: Razem się musimy trzymać, razem, my, artyści,\\
Motłoch sztuki nie rozumie, zagrajcie mi „Whisky”\\
(Moja żono!)\\
Na to, na dźwięk słowa „whisky” ledwo wstają rezerwiści:\\
Ej, ty, grajek, nie podskakuj, zagraj że co dla wojaków,\\
Bo jak nie – dostaniesz w ryj!

Ledwo żeśmy uszli z życiem, już do domu zdążam,\\
Patrzę, koleś psika sprayem, pisze: „Jebnie bomba”.\\
Gdy ujrzałem jego dzieło odżyła nadzieja,\\
Że ta bomba wreszcie jebnie albo ja was powystrzelam!

\end{textn}
\begin{chord}
    \footnotesize{
    A\\
    A\\
    A\\
    A\\
    A\\
    A\\
    A\\
    A\\
    A\\
    A\\
    A(CD)\\
    A\\
    A\\
    A\\
    A

    CD\\
    A\\
    CD\\
    FA
    }
\end{chord}