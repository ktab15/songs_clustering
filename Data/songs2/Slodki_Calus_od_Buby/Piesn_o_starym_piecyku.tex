%%
%% Author: bartek.rydz
%% 14.02.2019
%%
% Preamble
\tytul{Pieśń o starym piecyku}{sł. Jarosław ('Królik') Zajączkowski, J. Jacewicz}{Gdańska Formacja Szantowa}
\begin{text}
    Gdzie za sztormem huczy sztorm,\\
    Zasypany śniegiem świat,\\
    Gdzie gromada starych kutrów śpi\\
    Po zawietrznej stromych skał.\\
    Zimują tam wielorybnicy,\\
    Dziwni ludzie z różnych stron.\\
    Polarną nocą pieca blask\\
    Rozjaśnia w sercach zimny mrok.

    \vin Żelaznego pieca blask,\\
    \vin Wokół ludzi ciasny krąg,\\
    \vin Podła dziura, zapomniany świat,\\
    \vin Zimne ręce, wiatr i mrok.

    Czasem opowieść swoją snuje ktoś,\\
    Bywa, że fantazję ma,\\
    Gdy opowiada, nie domyślisz się,\\
    Gdzie prawda, a gdzie fałsz.\\
    O kopalniach, o dziewczynach,\\
    O kanionach pośród skał,\\
    A kiedy mówi, jego oczy\\
    Błyszczą w mroku zimnych ścian.

    Zimno, aż czasem chce się wyć,\\
    Czasem w myślach błądzi strach.\\
    Codzienny marsz, zmrożony śnieg,\\
    Lodowy deszcz, cholerny wiatr.\\
    I tak przez siedem dni w tygodniu,\\
    Smutne noce, whisky, grog,\\
    Czasami bójka, rozkwaszony nos,\\
    Kiepskie życie, gorzki los.

    Gdy skończy się polarna noc\\
    I nadejdzie długi dzień,\\
    Przypłyną kumple, będzie grog i śpiew,\\
    Znów popłynie w żyłach krew.\\
    Przeklętą dziurę wreszcie rzucę,\\
    Podłej budy trzasną drzwi,\\
    Zabiorę z sobą pieca blask,\\
    Który w sercu się tli.
\end{text}
\begin{chord}

\end{chord}