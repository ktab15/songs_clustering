%%\\
%% Author: bartek.rydz\\
%% 1^{7}.02.2019\\
%%\\
% Preamble\\
\tytul{Ze sceny}{sł. i muz. J. Kaczmarski wg W. Wysockiego 1977}{Jacek Kaczmarski}
\begin{text}
    Wy w ciemnościach –  reflektory chronią was –\\
    Oświetlając tylko scenę, na niej mnie!\\
    Jak na dłoni widać mą stężałą twarz,\\
    Gdy przez mrok próbują oczy przebić się!\\
    Mikrofony wychwytują każdy dźwięk,\\
    Mój najlżejszy oddech usłyszycie stąd!\\
    Ja przemogę sztywność zaciśniętych szczęk\\
    Wstrzymam myśli niekontrolowany prąd!

    Ja tu na krótko! Kochani – pozwolicie?\\
    Przed wami chcę naprawdę szczerze się wysilić!\\
    To dla was chwila, dla mnie całe życie!\\
    Nim zniknę – niech pokrzyczę krótką chwilę!

    Każdy chce mieć i każdy tak czy owak ma\\
    Tę chwilę krzyku między wejściem swym i wyjściem!\\
    Każdy na jakimś instrumencie gra,\\
    Choć nie każdego oklaskuje się rzęsiście.\\
    Lecz ja – ja wiem, ta krótka chwila długo trwa,\\
    Ale mam tyle, drodzy, wam do powiedzenia!\\
    Tylko przeszkadza mi ta za kurtyną twarz\\
    I ciągły szept, że to już koniec przedstawienia!

    Nie! Jeszcze trochę! Mamy czas! Bo widzicie –\\
    Do stracenia nikt z nas nie ma nic – i tyle\\
    A więc krzyczmy! Krótką chwilę – całe życie!\\
    Nim znikniemy – głośno krzyczmy krótką chwilę!

    Gwiżdże ktoś – nie mówię nic, nie było nic!\\
    Ja mikrofon mam i ja mam teraz głos!\\
    Tam za kulisami wy! Możecie iść!\\
    Przejmuję program i prowadzę dalej go!\\
    Nie przerywać! Tego, co warcholi – precz!\\
    Nie dla niego tutaj płuca w strzępy rwę!\\
    Hej, akustyk! Chrypnę! Gorzej słychać mnie!\\
    Silniej wzmacniacz! Głos mój teraz musi brzmieć!

    Rozciągnąć czas! Ej, wy tam w mroku – czy wierzycie\\
    Mnie, który wie jak się w epokę zmienia chwilę?\\
    Otwórzcie drzwi! I zaraz zobaczycie\\
    Jak marny czas, co gnębił nas, zostaje w tyle!

    Ktoś mi tutaj może powie, że już charczę,\\
    Że już nie rozróżniam poszczególnych słów,\\
    Ale mnie śpiewania jeszcze nie wystarczy!\\
    Jeśli przerwę - nigdy nie zaśpiewam znów!\\
    Nowych chwytów na gitarze nie wyćwiczę,\\
    Ale starych jeszcze dość – ja się nie mylę!\\
    Całe życie z sensem krzyczę, całe życie!\\
    Więc pokrzyczę jeszcze póki mam tę chwilę!

    Co się stało!? Czemu ten reflektor zgasł?\\
    Kto wyłączył mikrofony mi?\\
    Gdzie jesteście, kto stąd wyprowadził was?\\
    Kto zatrzasnął między nami drzwi?!\\
    Niech chociaż skończę! Bez puenty odchodzicie!\\
    A ma być śmieszna, najważniejsza, w wielkim stylu!\\
    I potrwa chwilę! Jedną w całym życiu!\\
    Nim zniknę... niech rozśmieszę was na chwilę...
\end{text}
\begin{chord}
    e a e\\
    E^{7} a\\
    a\\
    C H^{7}\\
    e a e\\
    E^{7} a\\
    a\\
    C H^{7}

    e a e\\
    E^{7} a\\
    a e\\
    C H^{7} e (a e)

    e a e\\
    E^{7} a\\
    a\\
    C H^{7}\\
    e a e\\
    E^{7} a\\
    a\\
    C H^{7}

    e a e\\
    E^{7} a\\
    a e\\
    C H^{7} e (a e)

    e a e\\
    E^{7} a\\
    a\\
    C H^{7}\\
    e a e\\
    E^{7} a\\
    a\\
    C H^{7}

    e a e\\
    E^{7} a\\
    a e\\
    C H^{7} e (a e)

    e a e\\
    E^{7} a\\
    a\\
    C H^{7}\\
    e a e\\
    E^{7} a\\
    a e\\
    C H^{7} e

    a e\\
    H C\\
    a e\\
    C H\\
    e H e (H^{7})\\
    e E a\\
    a e\\
    C H e
\end{chord}