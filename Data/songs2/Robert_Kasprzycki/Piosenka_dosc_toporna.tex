%%
%% Author: bartek.rydz
%% 28.05.2018
%%
% Preamble
\tytul{Piosenka dość toporna}{}{Robert Kasprzycki}
\begin{text}

    Niewinna owczą sierścią lecz w lędźwiach płomień chowasz\\
    Twych mlecznobiałych piersiąt niejeden chciał spróbować\\
    Lecz tylko ów literat co wielki ma dorobek\\
    Nagrody liczne zbiera i szczęście daje tobie\\
    Z nim dobre masz stosunki gdzieś w dużo wyższych sforach\\
    Stubarwne pijesz trunki z lśniącego termofora\\
    Ja tęsknię dziś tak bardzo ta miłość mnie pożera\\
    Choć inni tobą gardzą bo kwitnie twa kariera\\
    Agniecha ech Agniecha\\
    Gdy cię widzę ciągle myślę o twych grzechach\\
    Tych najcięższych oczywiście które ciążą zawiesiście\\
    Z tamtych czasów gdym bezmyślnie cię zaniechał\\
    Agniecha ech Agniecha\\
    Gdy cię widzę ciągle rośnie mi deprecha\\
    Na ulicach kamienicach patrzy na mnie twoja twarz\\
    Agniecha ty to szczęście masz\\
    Śpiewałaś kiedyś cienko bynajmniej nie sopranem\\
    Dziś dzielisz się sukienką z tym mydłkowatym panem\\
    Dla niego igła z nitką a dla mnie tylko guzik\\
    Gardzicie przyzwoitką bo już jesteście duzi\\
    On żonę okłamuje w twej ciasnej garsonierze\\
    Od dawna zachowuje się jak ostatnie zwierzę\\
    To on rządzi pilotem to on sypia od okna\\
    To jemu lśni szczoteczka od zębów jeszcze mokra\\
    Lecz wierzę że nadejdzie ten dzień gdy do mnie wrócisz\\
    W Paryżu albo w Lejdzie ofiarą mojej chuci\\
    Zostaniesz moja miła owieczko z krętym runem\\
    W swych delikatnych żyłach poczujesz mocny trunek\\
    Nic poza ostrzem brzytwy nie przyjdzie ci do głowy\\
    I już nie zdążysz spostrzec bez szkiełek kontaktowych\\
    Że lekką mam psychozę więc tylko w samych getrach\\
    W walizkach cię odwiozę na jakąś stację metra
    
\end{text}
\begin{chord}

\end{chord}