%%
%% Author: bartek.rydz
%% 28.05.2018
%%
% Preamble
\tytul{Kosovo II}{}{Robert Kasprzycki}
\begin{textn}

    Wiszą nad szarą ziemią pod lodowatym słońcem.\\
    Przecięte płomieniem, krzykiem ptaków płonące.\\
    Z bladą przekreską, czarną plamą, poręczą do nieba.\\
    Nie sięgnę już wyżej, nie będę już tam, nie pójdę z nimi.\\
    Schody donikąd. Płonąca drabina.\\
    Schody donikąd. Tam się zatrzymam.

    A teraz obraz się zmienia i bledną płomienie\\
    i ludzie w płomieniach. Ich światłocienie.\\
    I modlą się cienie do ziemi nisko świerszcze milczą i milczą obłoki.\\
    I widzę to. Tak zamknięty i sam. Tak bardzo sam, że tylko podaję sobie rękę.\\
    I teraz mam tylko tę szarą bezbronną piosenkę.\\
    Z przestrzeloną głową, trzecim okiem w pustej potylicy\\
    oglądam obłok tak blisko teraz tak blisko teraz. Tak blisko teraz\\
    Schody donikąd. Płonąca drabina.\\
    Schody donikąd. Tam się zatrzymam.

    Na kosowym polu gwiżdżą ptaki, czarne kosy z czarnym piórem płynące wysoko,\\
    to strzał czy obłok, kula czy wiatr tak tańczy w młodym lesie.\\
    A my z związanymi drutem dłońmi czujemy jak zmienia się powietrze\\
    i gaśnie już ogień, gdzie czterej żołnierze wpatrzeni w płomień płaczą.\\
    A my z koralikiem pod martwym językiem milczymy tak cicho. Tak blisko teraz.\\
    Schody donikąd. Płonąca drabina.\\
    Schody donikąd. Tam się zatrzymam.
    
\end{textn}
\begin{chord}
    \footnotesize{}
\end{chord}