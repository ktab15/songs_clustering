\tytul{Tawerna pod pijaną zgrają}{}{Grzegorz Bukała}
\begin{text}
    Kiedy niebo do morza przytula się z płaczem,\\
    Liche sosny garbate do reszty wykrzywia,\\
    Brzegiem morza wędrują bezdomni tułacze\\
    I nikt nie wie skąd idą, jaki wiatr ich przywiał...

    \vin Do tawerny Pod pijaną zgrają\\
    \vin Do tańczących, rozhukanych ścian,\\
    \vin I do dziewczyn, które serca za złamany grasz oddają,\\
    \vin Nie pytając czyś ty kiep, czy drań.

    Kiedy wiatry noc chmurną przegonią za wodę,\\
    Gdy pół-słońce pół nieba, pół morza rozpali,\\
    Opuszczają wędrowcy uśpioną gospodę,\\
    Z pierwszą bryzą znikają w pomarszczonej dali.

    \vin A w tawernie Pod pijaną zgrają\\
    \vin Spływa smutek z okopconych ścian,\\
    \vin A dziewczyny z półgrosików amulety układają,\\
    \vin Na kochanie, na tęsknotę i na żal.

    Kiedy chandra jesienna jak mgła cię otoczy,\\
    Kiedy wszystko postawisz na kartę przegraną,\\
    Zamiast siedzieć bezczynnie i płakać lub psioczyć,\\
    Weź węzełek na plecy, ruszaj w świat, w nieznane...

    \vin Do tawerny Pod pijaną zgrają\\
    \vin Do tańczących, rozhukanych ścian.\\
    \vin I do dziewczyn, które serca za złamany grosz oddają,\\
    \vin Nie pytając czyś ty kiep, czy drań.
\end{text}
\begin{chord}
    C E a F\\
    C G^0 d^7 G\\
    C E a F\\
    C F G C a

    F G^7 C a\\
    F G^7 C a\\
    F G C A^7\\
    F G C a\\
    (F G^7)
\end{chord}