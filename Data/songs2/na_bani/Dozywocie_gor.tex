\tytul{Dożywocie gór}{sł. Tomasz Borkowski, muz. B. Adamczak}{Na Bani}
\begin{text}
    Drogi Mistrzu - Mistrzu mej drogi\\
    Mistrzu Jerzy i Mistrzu Wojciechu\\
    Przez Was w górach schodziłem nogi\\
    Nie mogąc złapać oddechu

    Gór co stoją nigdy nie dogonię\\
    Znikających punktów na mapie\\
    Jakie miejsce nazwę swym domem\\
    Jakim dotrę do niego szlakiem?

    \hfill\break
    \vin Gór mi mało i trzeba mi więcej\\
    \vin Żeby przetrwać od zimy do zimy\\
    \vin Z wyrokiem wędrówki bez końca\\
    \vin Po śladach, które sam zostawiłem

    \vin Góry, góry i ciągle mi nie dość\\
    \vin Skazanemu na gór dożywocie\\
    \vin Świat na dobre mi zbieszczadział\\
    \vin Szczyty wolnym mijają mnie krokiem

    Pańscy święci - i święci bezpańscy\\
    Święty Jerzy, Mikołaju, Michale\\
    Opuszczeni gór świętych mieszkańcy\\
    Wasze imię pieśniami wychwalam

    Gór co stoją nigdy nie dogonię\\
    Znikających punktów na mapie\\
    Ani chaty, którą nazwałbym domem\\
    Gdzie żaden szlak by nie trafił

    \vin Gór mi mało i trzeba mi więcej\\
    \vin Żeby przetrwać od zimy do zimy\\
    \vin Z wyrokiem wędrówki bez końca\\
    \vin Po śladach, które sam zostawiłem

    \vin Góry, góry i ciągle mi nie dość\\
    \vin Skazanemu na gór dożywocie\\
    \vin Świat na dobre mi zbieszczadział\\
    \vin Szczyty wolnym mijają mnie krokiem

    \vin Gór mi mało i trzeba mi więcej\\
    \vin Abym przetrwał od zimy do zimy\\
    \vin Skazany na wieczną wędrówkę\\
    \vin Po śladach, które sam zostawiłem

    \vin Góry, góry i nigdy mi nie dość\\
    \vin Z gór dożywocia na karku wyrokiem\\
    \vin Świat na dobre mi zbieszczadział\\
    \vin Do szczytu wolnym zbliżam się krokiem
\end{text}
\begin{chord}
    d C G\\
    a G F\\
    d C G\\
    a G F

    d C G\\
    a G F\\
    d C G\\
    a G F\\
    d C e F C G\\
    dCG adE^4_7\\
    A^9 F C G\\
    A^9 F C G\\
    A^9 F C G\\
    A^9 F C G

    D e C D\\
    D e C D\\
    D e C D\\
    D e C D
\end{chord}