\tytul{Stare anioły}{sł. Tom Borkowski, muz. Izolda Trojanowska}{Na Bani}
\begin{text}

Wsparte na łasce boskiej chodzą po cmentarzach\\
Mimo bólu w krzyżu nie splamią się łzą.\\
To deszcz żłobi zmarszczki w ich, kamiennych twarzach,\\
Kiedy skrzydła niosą ukradkiem, na złom.

Zbyt duży to bagaż na barkach anioła,\\
Zwłaszcza, że brak młodzieńczego polotu.\\
Wstydliwy ciężar, pod płaszcz lepiej schować.\\
Skoro się go użyć nie miało, ochoty.

Bluźnią stare anioły, że nie były młode.\\
Że nigdy tak naprawdę latać nie umiały.\\
Że im całą wieczność, nie powstało w głowie\\
By na cudze piórka zamienić, przebranie.

A można było zagrać w serso aureolą.\\
Z kopyta ruszyć w dół, zawoławszy „Z Bogiem!”\\
W tyle pozostawić tylko kurzu ogon\\
I chichocząc zniknąć czym prędzej, za rogiem.

Anioły się krzywią w diabelnym grymasie\\
Gdy naiwni poeci, chwytają za pióra.\\
Wierząc, że to pióra nie ziemskie, nie ptasie.\\
I śmiertelnie poważni, wypisują bzdury.

Anioły mają za złe, że niebieskim kolorem\\
Że o wielkie nieba, w przyziemnym pisaniu.\\
Opierają się wiecznie łatwe metafory\\
I ponoć stąd, aniołów na wieki wieków lament
    
\end{text}
\begin{chord}
    \small{
    \textit{(Capo III)}\\
    F a\\
    G F\\
    a e\\
    G F e a

    a F G\\
    a F G\\
    a F G\\
    a F G\\
    (F d E)\\
    (a a/B a/C)\\
    (F^{7+} E)\\
    (a a/B a/C)\\
    (d e)
    }
\end{chord}