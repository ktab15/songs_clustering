\tytul{Ballada o Janie z Kolna}{muz. Mysza Michalska, sł. Tom Borkowski}{Na Bani}
\begin{text}

Żył na obcej mu ziemi, bo sam ziemi nie miał\\
Stałą ziemię miał za nic, a morze za wszystko\\
Gdy niełaska go pańska wygnała z bram Gdańska\\
Kaprem u króla Danii być - lub nie być - przyszło

W mrok pchała go bryza, gdy ścigał horyzont\\
Straszny nie był mu żaden Lewiatan ni Kraken\\
Z morzem brał się za bary, a na oceany\\
Wiódł go spienionym śladem Wikingów kilwater

Gdy świat jeszcze Bóg w wielkiej dłoni swej ważył\\
Jemu ziemia u nóg wydała się kulą\\
O rejsie w zaświaty ważył się marzyć\\
Gdzie królem był Kraken, za Ultimam Thulen

Chociaż na karawelę z nim wsiadło niewielu\\
Oni twardsze od lądu przemóc mogli morze\\
Każdy gotów żeglować pod góry lodowe\\
Aż północne im prądy ukazały zorze

Lecz gdy kursem zachodnim płynęły tygodnie\\
Mrozy skuły wspomnienie o ognistej Hekli\\
I nim jak sama radość z mórz wyrósł Labrador\\
Już po stokroć zechcieli swoją wachtę przekląć

Gdy świat jeszcze Bóg w wielkiej dłoni swej ważył\\
Komu Ziemia u nóg zaczęła być kulą\\
O rejsie w zaświaty kto ważył się marzyć\\
Gdzie królem Lewiatan, za Ultimam Thulen

Ku krawędzi wszechświata, na życia krawędzi\\
Żeglarz Jan płynął odkryć granice – dla siebie\\
Cóż, gdy świat nadal płaski wciąż wolał trwać w błędzie\\
W służbie świata Janowi przyszło być – albo nie być
 
\end{text}
\begin{chord}
    \small{
    \textit{Cap II}\\
    a e C G\\
    C G a e a\\
    a e C G\\
    C G a e a

    a e C G\\
    C G a e a\\
    a e C G\\
    C G a e a

    a G a\\
    a G a\\
    a G a\\
    a e a
    }
\end{chord}