%%
%% Author: bartek.rydz
%% 28.05.2018
%%
% Preamble
\tytul{Song o ciszy}{}{Czerwony Tulipan}
\begin{textn}

    Wy mnie słuchacie, a ja śpiewam tekst z muzyką.\\
    Taka konwencja, taki moment, więc tak jest.\\
    Zaufaliśmy obyczajom i nawykom,\\
    już nie pytamy czy w tym wszystkim jakiś sens.\\
    A ja zaśpiewać dzisiaj chcę w obronie ciszy,\\
    choć wiem: nie pora, nie miejsce i nie czas.

    Lecz gdy się milczy, milczy, milczy to apetyt rośnie wilczy\\
    na poezję, co być może drzemie w nas.\\
    Bo gdy się milczy, milczy, milczy to apetyt rośnie wilczy\\
    na poezję, co być może drzemie w nas.

    Przecież dosyć już mamy huku i jazgotu.\\
    Ale gdy cicho, to źle i głupio nam,\\
    jakby się zepsuł życia niezawodny motor,\\
    coś nie w porządku jakbyś był już nie ten sam.\\
    Cisza zagłusza, sam już nie wiesz jaki jesteś,\\
    więc szybko włączasz wszystko co pod ręką masz.

    Gdy kiedyś nagle łomot umrze w dyskotekach,\\
    do siebie nam dalej będzie niż do gwiazd.\\
    Zanim coś powiesz tak jak człowiek do człowieka,\\
    cisza zgruchocze i wykrwawi wszystkich nas.\\
    Dlatego uczmy się ciszy i milczenia.\\
    To siostry myśli świadomości przednia straż.

\end{textn}
\begin{chord}
    \small{
        \textit{Capo II}\\
        A a\\
        A C\\
        A C D A\\
        A C D E\\
        A e\\
        A e

        a e\\
        H^{7} e\\
        a e\\
        H^{7} e

        A a\\
        A C\\
        A C D A\\
        A C D E\\
        A e\\
        A e

        A a\\
        A C\\
        A C D A\\
        A C D E\\
        A e\\
        A e
    }
\end{chord}