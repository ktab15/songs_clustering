%%
%% Author: bartek.rydz
%% 10.02.2019
%%
% Preamble
\tytul{Szanta narciarska}{Artur Andrus}{Artur Andrus}
\begin{textn}

    Nazywali go marynarz, bo opaskę miał na oku.\\
    Na każdym stoku dziewczyna, dziewczyna na każdym stoku.\\
    Pochodzi spod Poznania, podobno umie wróżyć z kart.\\
    Panny rwie na wiązania, mężatki - na długość nart.

    Caryco mokrego śniegu ratrakiem płynę do Ciebie pod prąd.\\
    Dobrze, że stoisz na brzegu, bo ja właśnie schodzę na ląd.

    Nigdy się nie lękał biedy i się nie przejmował jutrem.\\
    A jego ratrak był kiedyś zwyczajnym rybackim kutrem.\\
    I woził dorsze i śledzie, zimą i latem, okrągły rok.\\
    Teraz jak nieraz przejedzie rybami czuć cały stok.

    Wszyscy w porcie odetchnęli zwiał nim się zakończył sezon.\\
    Jeszcze się tam jak żagiel bieli jego czarny kombinezon.\\
    Odpłynął pod Ustrzyki i przez kobiety wpadł w kłopoty.\\
    Forsę z polowań na orczyki przehulał na antybiotyk.

    Jeśli kiedyś go zobaczysz na ratraku w podłym świecie,\\
    To powiedz mu, że w Karpaczu czekają na niego dzieci.\\
    I kiedy opuszcza statek, żeby się znowu oddać złu,\\
    Każda z dwudziestu siedmiu matek dzieciątku śpiewa do snu:

\end{textn}
\begin{chord}
    \ifOneCol \else \footnotesize{ \fi
    \textit{Cap II}\\
    d a d C F\\
    g d B a d\\
    d a d C F\\
    g d B a d

    d a d g\\
    g d B a d
    \ifOneCol \else } \fi
\end{chord}