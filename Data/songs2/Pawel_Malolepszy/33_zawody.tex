%%
%% Author: EL Profesor
%% %26.09.2020
%%
\tytul{33 Zawody}{sł. i muz. Marek Kojro}{Grzmiąca Półlitrówa}
\begin{text}
    
    Spójrz na dom ten za zielonym wzgórzem\\
    Tam do niedawna był ten pusty plac\\
    Przy budowie jego stary majster\\
    Uczył mnie, jak cegły kłaść\\
    Nie zapomnę nigdy tamtych dni\\
    Gdy na głowę z nieba lał się wielki żar\\
    Do dziś jeszcze od roboty tej\\
    Zza paznokci wydłubuję piach

    \vin Trzydzieści trzy zawody\\
    \vin dwadzieścia parę lat\\
    \vin I ciągle czegoś szukam\\
    \vin i ciągle czegoś mi brak

    Znowu pytasz, chociaż dobrze wiesz\\
    Że ja wszystkich na południu dobrze znam\\
    Na tym zdjęciu ciężarówka stara\\
    Obok brudas ten to właśnie ja\\
    Pomagałem ludziom ile sił\\
    Przewoziłem swą staruszką co się da\\
    Do dziś jeszcze od roboty tej\\
    Zza paznokci wydłubuję smar

    Facet ten, co przed chwilą właśnie\\
    Kłaniał nam się tak z szacunkiem w pas\\
    Zwiedził ze mną kiedyś dawno temu\\
    Świata tego całkiem niezły szmat\\
    Pieściliśmy swoją ciuchcię\\
    Byle tylko gnała dalej, byle w przód\\
    Do dziś jeszcze od roboty tej\\
    Gra mi w głowie stukot kół

    Śmiejesz się, ale znam te góry\\
    Tam przyjaciół niezłych jeszcze kilku mam\\
    Nie uwierzysz, kiedy powiem tobie\\
    Że był kiedyś ze mnie niezły drwal\\
    Harowałem ciężko cały dzień\\
    By wieczorem przy ognisku wódę chlać\\
    Do dziś jeszcze od roboty tej\\
    Ręce po kolana mam

    A pamiętasz, kiedy ja stukałem\\
    Do drzwi Twych przez miesiące chyba trzy\\
    I codziennie dziwnie uśmiechnięta\\
    Odbierałaś z dłoni mojej list\\
    Sam już nie wiem, jak to było\\
    Kiedy zamiast listu miałem w dłoni kwiat\\
    Do dziś jeszcze od roboty tej\\
    To klnę jak szewc na cały świat

	
\end{text}
\begin{chord}
    C\\
    F C\\
    C\\
    C G\\
    C\\
    F C\\
    C\\
    G G^7 C

    C\\
    F C\\
    F\\
    C G C

    C\\
    F C\\
    C\\
    C G\\
    C\\
    F C\\
    C\\
    G G^7 C

    C\\
    F C\\
    C\\
    C G\\
    C\\
    F C\\
    C\\
    G G^7 C

    C\\
    F C\\
    C\\
    C G\\
    C\\
    F C\\
    C\\
    G G^7 C

    C\\
    F C\\
    C\\
    C G\\
    C\\
    F C\\
    C\\
    G G^7 C

\end{chord}