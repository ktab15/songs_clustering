%%
%% Author: EL PROFESOR
%% %28.09.2020
%%
\tytul{Na Mazury}{sł. i muz. Lucjusz Michał Kowalczyk}{Spinakery}
\begin{text}
    Się masz, witam cię, piękną sprawę mam!\\
    Pakuje bety swe i leć ze mną w dal.\\
    Rzuć kłopotów stos, no nie wykręcaj się,\\
    Całuj wszystko w nos,\\
    Osobowym drugą klasą przejedziemy się.

    \vin Na Mazury, Mazury, Mazury,\\
    \vin Popływamy tą łajbą z tektury,\\
    \vin Na Mazury, gdzie wiatr zimny wieje,\\
    \vin Gdzie są ryby i grzyby, i knieje.

    \vin Tam gdzie fale nas bujają,\\
    \vin Gdzie się ludzie opalają,\\
    \vin Wschody słońca piękne są\\
    \vin I komary w dupę tną,

    \vin Gdzie przy ogniu gra muzyka\\
    \vin I gorzała w gardle znika.\\
    \vin Pan leśniczy, nie wiadomo skąd,\\
    \vin Woła do nas Ooo!... paszła won!

    Uszczelniłem dno, lekko chodzi miecz,\\
    Zezy smrodów sto przewietrzyłem precz,\\
    Ster nie spada już i grot luzuje się,\\
    Więc się ze mną rusz,\\
    Już nie będzie tak jak wtedy, nie denerwuj się.

    \vin Na Mazury, Mazury, Mazury,\\
    \vin Popływamy tą łajbą z tektury,\\
    \vin Na Mazury, gdzie wiatr zimny wieje,\\
    \vin Gdzie są ryby i grzyby, i knieje.

    \vin Tam gdzie fale nas bujają,\\
    \vin Gdzie się ludzie opalają,\\
    \vin Wschody słońca piękne są\\
    \vin I komary w dupę tną,

    \vin Gdzie przy ogniu gra muzyka\\
    \vin I gorzała w gardle znika.\\
    \vin Pan leśniczy, nie wiadomo skąd,\\
    \vin Woła do nas Oooo!... paszła won!

    Skrzynkę piwa mam, ty otwieracz weź,\\
    Napić ci się dam tylko mi ją nieś!\\
    Coś rozdziawił dziób i masz głupi wzrok,\\
    No nie stój jak ten słup!\\
    Z Węgorzewa na Ruciane wykonamy skok!

    \vin Na Mazury, Mazury, Mazury,\\
    \vin Popływamy tą łajbą z tektury,\\
    \vin Na Mazury, gdzie wiatr zimny wieje,\\
    \vin Gdzie są ryby i grzyby, i knieje.

    \vin Tam gdzie fale nas bujają,\\
    \vin Gdzie się ludzie opalają,\\
    \vin Wschody słońca piękne są\\
    \vin I komary w dupę tną,

    \vin Gdzie przy ogniu gra muzyka\\
    \vin I gorzała w gardle znika.\\
    \vin Pan leśniczy, nie wiadomo skąd,\\
    \vin Woła do nas Ooo!... dzień dobry żeglarze!
\end{text}
\begin{chord}
    C G C\\
    C G C\\
    F C G C\\
    F C\\
    G

    C F C\\
    C G C\\
    C F C\\
    C G C

    F C\\
    G C\\
    F C\\
    G C

    F C\\
    G C\\
    F C\\
    G C

    C G C\\
    C G C\\
    F C G C\\
    F C\\
    G

    C F C\\
    C G C\\
    C F C\\
    C G C

    F C\\
    G C\\
    F C\\
    G C

    F C\\
    G C\\
    F C\\
    G C
	
	C G C\\
    C G C\\
    F C G C\\
    F C\\
    G

    D G D\\
    D A D\\
    D G D\\
    D A D

    G D\\
    A D\\
    G D\\
    A D

    G D\\
    A D\\
    G D\\
    A D
\end{chord}