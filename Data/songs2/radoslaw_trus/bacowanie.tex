\tytul{Bacowanie}{sł. Radosław Truś}{}
\begin{text}
    Późną nocą, nie wiem, po co, ludzie się pod górę pocą,\\
    ciężki oddech, ciężki plecak,\\
    jedno tylko ich podnieca:\\
    Tam wysoko, na polanie,\\
    będzie piękne bacowanie.\\
    \vin Bacowanie, bacowanie, to prawdziwy życia raj,\\
    \vin nigdy nie wiesz, co się stanie,\\
    \vin wietrze wiej, wietrze wiej\\
    \vin wietrze graj!

    A na hali domek z bali, nawet Tatry widać z dali,\\
    kryta gontem kurna chata,\\
    w środku liście wiatr zamiata,\\
    W lecie – mówiąc między nami\\
    – tam juhasi śpią z owcami.\\
    \vin Bacowanie, bacowanie...

    Chęci szczere, więc siekierę w obie dłonie mocno bierę,\\
    zbieram siły me w dwójnasób\\
    trzeba drewno przynieść z lasu.\\
    Potem się przyjrzałem dobrze:\\
    zamiast suszki ściąłem modrzew.\\
    \vin Bacowanie, bacowanie...

    Do źródliska, gdzie zdrój tryska, przez wykroty i urwiska,\\
    czysta woda zdrowia doda,\\
    warg wyschniętych to ochłoda.\\
    Lecz się nagle robię słaby:\\
    w wodzie się kochają żaby.\\
    \vin Bacowanie, bacowanie...

    A więc pulpa, mea culpa, będzie słona, bo mi sól tam\\
    cała wpadła zamiast ryżu\\
    tamta puszka była niżej…\\
    Doświadczenie moją wiedzą:\\
    jak są głodni, wszystko zjedzą.\\
    \vin Bacowanie, bacowanie...

    Watra płonie, wszyscy do niej przemarznięte zbliżą dłonie,\\
    na gitarze zagram pięknie\\
    aż wam wszystkim serce pęknie.\\
    Gdy już każdy się zadumał,\\
    zamiast serca pękła struna.\\
    \vin Bacowanie, bacowanie...
\end{text}
\begin{chord}
    C\\
    d G^7 d G^7\\
    d G^7\\
    F G C a\\
    D G\\
    C F G F G\\
    C\\
    F G F G\\
    D G

    C\\
    d G^7 d G^7\\
    d G^7\\
    F G C a\\
    D G\\

    C\\
    d G^7 d G^7\\
    d G^7\\
    F G C a\\
    D G\\

    C\\
    d G^7 d G^7\\
    d G^7\\
    F G C a\\
    D G\\

    C\\
    d G^7 d G^7\\
    d G^7\\
    F G C a\\
    D G\\

    C\\
    d G^7 d G^7\\
    d G^7\\
    F G C a\\
    D G\\
\end{chord}
