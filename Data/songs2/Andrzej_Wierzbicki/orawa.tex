\tytul{Orawa}{sł. i muz. Andrzej Wierzbicki}{}
\begin{text}
    Z mego okna widać chmury na skalistych grzędach,\\
    Przetrę szybę ciepłą dłonią, razem z nimi siędę,\\
    I będą mi grały wiatry na organach turni,\\
    Kiedy pójdę zbójnikować nad dachami równin.

    Z mego okna widać potok – doliną, doliną,\\
    Dumnych smreków las szeroki, mgły w kosodrzewinach,\\
    I będą mi grały wiatry w zaklętych kolebach,\\
    Noc krzesanym się roztańczy po niebach, po niebach...

    \vin Orawo! Wiatrem malowany dach,\\
    \vin Ciupagami wysrebrzany na smrekowych pniach.\\
    \vin Orawo! Wiatrem malowany dom,\\
    \vin Gdzie zbójnickie śpiewogrania po kolebach śpią.

    Z mego okna widać chmury na skalistych grzędach,\\
    Przetrę szybę ciepłą dłonią, razem z nimi siędę,\\
    I będą mi grały wiatry na organach turni,\\
    Moje życie tylko w górach, nad dachami równin.

    \vin Orawo! Wiatrem malowany dach...
\end{text}
\begin{chord}
    a C d E\\
    a C d E\\
    F C d E\\
    a C d E a

    a C d E\\
    a C d E\\
    F C d E\\
    a C d E a

    F C d E\\
    C G H^7 E E^7\\
    F C d E\\
    F H^7 E
\end{chord}